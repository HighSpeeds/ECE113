\documentclass[12pt]{article}
\author{Lawrence Liu}
\usepackage{subcaption}
\usepackage{graphicx}
\usepackage{amsmath}
\usepackage{pdfpages}
\newcommand{\Laplace}{\mathscr{L}}
\setlength{\parskip}{\baselineskip}%
\setlength{\parindent}{0pt}%
\usepackage{xcolor}
\usepackage{listings}
\definecolor{backcolour}{rgb}{0.95,0.95,0.92}
\usepackage{amssymb}
\lstdefinestyle{mystyle}{
    backgroundcolor=\color{backcolour}}
\lstset{style=mystyle}

\title{ECE 113 HW 6}
\begin{document}
\maketitle
\section*{Problem 1}
\subsection*{(a)}
We have $10000\cdot 5$ samples for our 
time domain signal, each of these will be $4$ bytes, 
so in total we would have that we would need 200kB to 
naively represent the signal. For our 
DFT signal we would have $10000 \cdot 5$ samples, each of
which would take $8$ bytes to represent, so in total 
we would need 400kB to represent the signal.
\subsection*{(b)}
Since it is a real value signal we would have that 
$$X_k=X_{N-k}^*$$
Thus we can only send half of the DFT signal,
which would be $10000 \cdot 5/2+1$ samples, each of which
would take $8$ bytes to represent, so in total we would
need 200kB to represent the signal.
\subsection*{(c)}
Then we would have that we would only 
need to send the values of 
$X_k$ between $k=2000$ and $k=4000$, which would be
$2001$ samples, each of which would take $8$ bytes to
represent, so in total we would need 16008 bytes to
represent the signal.
\section*{Problem 2}
\subsection*{(a)}
From nyquist we would need to sample at least $2\cdot 8000$ Hz, 
so then we would need $2\cdot 8000\cdot 5=80000$ bytes to represent
the signal in discrete time. Likewise we would need $80001$ bytes 
to represent the signal in the frequency domain since the signal is discrete.
\subsection*{(b)}
We only need to 
sample at 2 times the bandwith which in this case is 
$4000$ Hz, so then we would could sample at $8000$ Hz and still reconstruct the signal.
\section*{Problem 3}
\subsection*{(a)}
We need to multiply every term by every term in the polynomial and summing 
so for two polynomials of degree $n$ and $m$, it will take 
$O(nm)$ operations to multiply them.
\subsection*{(b)}
We have that given two polynomials $a(x)=\sum_{k=0}^{n-1}a_kx^k$ and
$b(x)=\sum_{k=0}^{m-1}b_kx^k$, then we have that
$$y(x)=a(x)b(x)=\sum_{i=1}^{n-1}\sum_{j=1}^{m-1}a_ix^ib_jx^j$$
$$y(x)=a(x)b(x)=\sum_{k=1}^{n+m-1}\sum_{j=1}^{n-1}a_{k-j}b_jx^k$$
Thus we get that the the coefficents of $y$, $y[k]$ is 
$$y[k]=a'\bigotimes b'$$
Where $a'$ and $b'$ are the coefficents of $a$ and $b$ respectively
as an array and then zero padded with $n+1$ zeros, 
and $y'$ is the circular convolution of $a'$ and $b'$.
\subsection*{(c)}
Therefore we can do it with FFT by first taking the FFT
of the signals made up of the coefficents of the two polynomials, 
then multiplying the two FFTs, and then taking the inverse FFT of the
result. This will take $O(n\log n)$ time.
\section*{Problem 4}
\subsection*{(a)}
Let 
$x(n)\to X(k)$, 
from the modulation property we have that
$$x(n)\cos(\frac{\pi}{2}n)=\frac{1}{2}\left(X(k-1)+X(k+1)\right)$$
$$x(n)\cos(\frac{\pi}{2}n)\to [1,\frac{1}{2},1,\frac{1}{2}]$$
\subsection*{(b)}
$$[0,0,1,0]\to[1,-1,1,-1]$$
Applying property that convolution in frequency domain is multiplication in time domain, we get that
$$[0,0,1,0]\circ x[n]\to [0,j-1,1,-1-j]$$
\subsection*{(c)}
$$[0,0,1,0]\to[1,-1,1,-1]$$
Applying property that convolution in frequency domain is multiplication in time domain, we get that
$$g[n]x[n]\to \frac{1}{4}[-1,1,-1,1]$$
\subsection*{(d)}
Applying the time shift property we get
$$x[n-1]\to e^{-j\frac{\pi}{2}}DFT(x[n])=[0,-j-1,-1,j-1]$$
\section*{Problem 5}
\subsection*{(a)}
We have that the z transform is 
$$X(z)=\sum_{n=-\infty}^{\infty}x(n)z^{-n}=\sum_{n=0}^{\infty}a^{n}z^{-n}$$
We have that this exists when $|az^{-1}|<1$, so we have that
the ROC is $|z|>|a|$, and if this exists we have that 
the z transform is:
$$X(z)=\frac{1}{1-a^{n}z^{-1}}$$
Since the ROC is $|z|>|a|$, and since the 
dft is $z=e^{j\omega}$, we have that the DFT exists if 
$1>|a|$
\subsection*{(b)}
We have that the z transform is
$$X(z)=\sum_{n=-\infty}^{\infty}x(n)z^{-n}=a\sum_{n=0}^{\infty}z^{-n}$$
We have that this exists when $|z|>1$, so we have that
the ROC is $|z|>1$, and if this exists we have that
the z transform is:
$$X(z)=\frac{a}{1-z^{-1}}$$
Since the ROC is $|z|>1$, and since the
dft is $z=e^{j\omega}$, we have that the  will not
exist since $|e^{j\omega}|=1$ for all $\omega$.

\end{document}