\documentclass[12pt]{article}
\author{Lawrence Liu}
\usepackage{subcaption}
\usepackage{graphicx}
\usepackage{amsmath}
\usepackage{pdfpages}
\newcommand{\Laplace}{\mathscr{L}}
\setlength{\parskip}{\baselineskip}%
\setlength{\parindent}{0pt}%
\usepackage{xcolor}
\usepackage{listings}
\definecolor{backcolour}{rgb}{0.95,0.95,0.92}
\usepackage{amssymb}
\lstdefinestyle{mystyle}{
    backgroundcolor=\color{backcolour}}
\lstset{style=mystyle}

\title{ECE 113 HW 6}
\begin{document}
\maketitle
\section*{Problem 1}
\subsection*{(a)}
We have $10000\cdot 5$ samples for our 
time domain signal, each of these will be $4$ bytes, 
so in total we would have that we would need 200kB to 
naively represent the signal. For our 
DFT signal we would have $10000 \cdot 5$ samples, each of
which would take $8$ bytes to represent, so in total 
we would need 400kB to represent the signal.
\subsection*{(b)}
Since it is a real value signal we would have that 
$$X_k=X_{N-k}^*$$
Thus we can only send half of the DFT signal,
which would be $10000 \cdot 5/2+1$ samples, each of which
would take $8$ bytes to represent, so in total we would
need 200kB to represent the signal.
\subsection*{(c)}
Then we would have that we would only 
need to send the values of 
$X_k$ between $k=2000$ and $k=4000$, which would be
$2001$ samples, each of which would take $8$ bytes to
represent, so in total we would need 16008 bytes to
represent the signal.
\section*{Problem 2}
\subsection*{(a)}
From nyquist we would need to sample at least $2\cdot 8000$ Hz, 
so then we would need $2\cdot 8000\cdot 5=80000$ bytes to represent
the signal in discrete time. Likewise we would need $80001$ bytes 
to represent the signal in the frequency domain since the signal is discrete.
\subsection*{(b)}
We have that the sampled signal's frequency domain is 
$$X_s=\sum_{k=-\infty}^{k=\infty}X(f-kf_s)$$
Where $X(f)$ is the frequency domain of the original signal, then 
since we have that $X(f)>0$ only for $f\in[4000,8000]$, 
then we get that if we just apply a  bandpass filter, we 
can sample at $4000$ Hz and still reconstruct the signal. so 
therefore we would only need $4000\cdot 8=32000$ bytes to represent
the signal in discrete time.
\section*{Problem 3}
\subsection*{(a)}
We need to multiply every term by every term in the polynomial and summing 
so it will take $O(n^2)$ time.
\subsection*{(b)}
We have that the coefficeint of $x^k$, $c_k$ is 
$$c_k=\sum_{i=0}^{i=k}a_ib_{k-i}$$
Where $a_i$ and $b_i$ are the coefficeints of the polynomials $a$ and $b$ for 
$x^i$. So we can see that $c_k$ is the circular convolution of $a$ and $b$.
\subsection*{(c)}
Therefore we can do it with FFT by first taking the FFT
of the signals made up of the coefficents of the two polynomials, 
then multiplying the two FFTs, and then taking the inverse FFT of the
result. This will take $O(n\log n)$ time.
\section*{Problem 4}
\subsection*{(a)}
We have that 
\subsection*{(b)}
$$[0,0,1,0]\to[1,j,-1,-j]$$
Applying property that convolution in frequency domain is multiplication in time domain, we get that
$$[0,1,0,0]\circ x[n]\to [0,1-j,-1,1-j]$$
\subsection*{(c)}
$$[0,0,1,0]\to[1,-j,-1,j]$$
\subsection*{(d)}
Applying the time shift property we get
$$x[n-1]\to e^{-j\frac{\pi}{2}}DFT(x[n])=[0,-j-1,-1,j-1]$$
\section*{Problem 5}


\end{document}