\include{"../preamble.tex"}
\title{ECE 113 HW 1}
\begin{document}
\maketitle
\section*{Problem 1}
\subsection*{(a)}
\includegraphics[scale=0.5]{c.png}
\subsection*{(b)}
\includegraphics[scale=0.5]{d.png}
\subsection*{(c)}
\includegraphics[scale=0.5]{e.png}
\subsection*{(d)}
\includegraphics[scale=0.5]{u.png}
\subsection*{(e)}
\includegraphics[scale=0.5]{v.png}
\subsection*{(f)}
\includegraphics[scale=0.5]{s.png}
\subsection*{(g)}
\includegraphics[scale=0.5]{r.png}
\section*{Problem 2}
\subsection*{(a)}
$$\boxed{20}$$
\subsection*{(b)}
$$\boxed{25}$$
\subsection*{(c)}
$$\boxed{40}$$
\subsection*{(d)}
$$\boxed{80}$$
\subsection*{(e)}
$$\boxed{20}$$
\subsection*{(f)}
$$\boxed{8}$$
\section*{Problem 3}
\subsection*{(a)}
Since $e^{j\theta} = \cos(\theta) + j\sin(\theta)$, we have
that the fundamental period of $\hat{x}_a[n]$ is $\boxed{8}$
\subsection*{(b)}
We have that $F_0=0.3$, therefore we have that the period of the sequence
is $\frac{k}{0.3}$. The minimum $k$ such that this evaluates to a
positive integer is $k=3$. Therefore, the fundamental period of
the sequence is $\boxed{10}$.
\subsection*{(c)}
The period of $e^{j\pi n /8}$ is $16$ and the period of
 $e^{j\pi n /5}$ is 10. Therefore, the period of the sequence
 is the least common multiple of these two, which is $\boxed{80}$. 
\subsection*{(d)}
The period of $\sin(0.15\pi n)$ is $40$ and the period of
 $\cos(0.12\pi n+0.1\pi)$ is 50. Therefore, the period of the sequence
 is the least common multiple of these two, which is $\boxed{200}$. 
\subsection*{(e)}
The period of $\sin(0.15\pi n+0.75\pi)$ is $10$ and the period of
 $\cos(0.8\pi n+0.2\pi)$ is 5, and the period of $\cos(1.3\pi n)$ is 20. Therefore, the period of the sequence
 is the least common multiple of these three, which is $\boxed{20}$.
 \section*{Problem 4}
 \subsection*{(i)}
 In order for the sequence to be periodic, we must have that
 $$x(1-2(n+N_2)) = x(1-2n)$$
 for some period $N_2$, since the signal is periodic for period of 
 $N$ we have that
 $$x(1-2n)=x(1-2n-N)=x(1-2n-2N)$$
 Therefore we have that $x(1-2(n))$ is periodic with period of $N_2=N$.
\subsection*{(ii)}
If $N$ is an integer then the signal is periodic, since $(-1)^{n+N}=(-1)^{n}$
if $N$ is even, and $(-1)^{n+2N}=(-1)^{n}$ if $N$ is odd. But if 
$N$ is not an integer, then the signal is not periodic.






\end{document}
