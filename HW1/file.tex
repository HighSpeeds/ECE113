\include{"../preamble.tex"}
\title{ECE 113 HW 1}
\begin{document}
\maketitle
\section*{Problem 1}
\subsection*{(a)}
$$\boxed{c[n]=\{2, 0, -1, 6, -3, 2\}, -6\leq n \leq-1}$$ 
the sample values for the sequence outside of the range specified
is 0.\\
\includegraphics[scale=0.5]{c.png}
\subsection*{(b)}
$$\boxed{d[n]=\{8, 2, -7, -3, 0, 1, 1\}, -3\leq n \leq3}$$ 
the sample values for the sequence outside of the range specified
is 0.\\
\includegraphics[scale=0.5]{d.png}
\subsection*{(c)}
$$\boxed{e[n]=\{2, -3, 6, -1, 0, 2\}, -2\leq n \leq3}$$ 
the sample values for the sequence outside of the range specified
is 0.\\
\includegraphics[scale=0.5]{e.png}
\subsection*{(d)}
$$\boxed{u[n]=\{8, 2, -7, -3, 0, 1, 1, 0, 2, 0, -1, 6, -3, 2\}, 
-8\leq n \leq5}$$ 
the sample values for the sequence outside of the range specified
is 0.\\
\includegraphics[scale=0.5]{u.png}
\subsection*{(e)}
$$\boxed{v[n]=\{-8, 4, -42, -18\}, 
-2\leq n \leq1}$$ 
the sample values for the sequence outside of the range specified
is 0.\\
\includegraphics[scale=0.5]{v.png}
\subsection*{(f)}
$$\boxed{s[n]=\{8, 2, -7, -3, 0, 1, 1, 0, 0, 0, -3, -6, 1, -2, -6, -6, -1\}, 
-9\leq n \leq7}$$ 
the sample values for the sequence outside of the range specified
is 0.\\
\includegraphics[scale=0.5]{s.png}
\subsection*{(g)}
$$\boxed{u[n]=\{11.7, 23.4, -3.9, 7.8, 23.4, 23.4, 3.9\}, 
-2\leq n \leq4}$$ 
the sample values for the sequence outside of the range specified
is 0.\\
\includegraphics[scale=0.5]{r.png}
\section*{Problem 2}
\subsection*{(a)}
$$\boxed{20}$$
\subsection*{(b)}
$$\boxed{25}$$
\subsection*{(c)}
$$\boxed{40}$$
\subsection*{(d)}
$$\boxed{80}$$
\subsection*{(e)}
$$\boxed{20}$$
\subsection*{(f)}
$$\boxed{8}$$
\section*{Problem 3}
\subsection*{(a)}
Since $e^{j\theta} = \cos(\theta) + j\sin(\theta)$, we have
that the fundamental period of $\hat{x}_a[n]$ is $\boxed{8}$
\subsection*{(b)}
We have that $F_0=0.3$, therefore we have that the period of the sequence
is $\frac{k}{0.3}$. The minimum $k$ such that this evaluates to a
positive integer is $k=3$. Therefore, the fundamental period of
the sequence is $\boxed{10}$.
\subsection*{(c)}
The period of $e^{j\pi n /8}$ is $16$ and the period of
 $e^{j\pi n /5}$ is 10. Therefore, the period of the sequence
 is the least common multiple of these two, which is $\boxed{80}$. 
\subsection*{(d)}
The period of $\sin(0.15\pi n)$ is $40$ and the period of
 $\cos(0.12\pi n+0.1\pi)$ is 50. Therefore, the period of the sequence
 is the least common multiple of these two, which is $\boxed{200}$. 
\subsection*{(e)}
The period of $\sin(0.1\pi n+0.75\pi)$ is $20$ and the period of
 $\cos(0.8\pi n+0.2\pi)$ is 5, and the period of $\cos(1.3\pi n)$ is 20. Therefore, the period of the sequence
 is the least common multiple of these three, which is $\boxed{20}$.
 \section*{Problem 4}
 \subsection*{(i)}
 In order for the sequence to be periodic, we must have that
 $$x(1-2(n+N_2)) = x(1-2n)$$
 for some period $N_2$, since the signal is periodic for period of 
 $N$ we have that
 $$x(1-2n)=x(1-2n-N)=x(1-2n-2N)$$
 Therefore we have that $x(1-2(n))$ is periodic with period of $N_2=N$.
\subsection*{(ii)}
If $N$ is an integer then the signal is $\boxed{\text{periodic}}$, since $(-1)^{n+N}=(-1)^{n}$, therefore
the period would be $\boxed{N}$
if $N$ is even, and $(-1)^{n+2N}=(-1)^{n}$, and thus the period would be $\boxed{2N}$ if $N$ is odd.
\section*{Problem 5}
With the following matlab code:
\lstinputlisting[
    basicstyle=\tiny, %or \small or \footnotesize etc.
]{problem5.m}
We get the following plots\\
\includegraphics[scale=0.2]{f_03.png}\\
\includegraphics[scale=0.2]{f_07.png}\\
\includegraphics[scale=0.2]{f_013.png}\\
As one can see, we can reconstruct the original conitnous time function from the first sample, since the sampling rate is greater than the nyquist sampling frequency, but not for the other two, since the sampling frequncy is less than the nyquist sampling frequency.




\end{document}
