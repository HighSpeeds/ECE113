\documentclass[12pt]{article}
\author{Lawrence Liu}
\usepackage{subcaption}
\usepackage{graphicx}
\usepackage{amsmath}
\usepackage{pdfpages}
\newcommand{\Laplace}{\mathscr{L}}
\setlength{\parskip}{\baselineskip}%
\setlength{\parindent}{0pt}%
\usepackage{xcolor}
\usepackage{listings}
\definecolor{backcolour}{rgb}{0.95,0.95,0.92}
\usepackage{amssymb}
\lstdefinestyle{mystyle}{
    backgroundcolor=\color{backcolour}}
\lstset{style=mystyle}

\title{ECE 113 HW 2}
\begin{document}
\maketitle
\section*{Problem 1}
\subsection*{(a)}
$$x_{1_{even}}[n]=\frac{u(n-3)+u(-(n+3))}{2}$$
$$x_{1_{odd}}[n]=\frac{u(n-3)-u(-(n+3))}{2}$$
\subsection*{(b)}
$$x_{2_{even}}[n]=\frac{\alpha^nu[n-1]+\alpha^{-n}u[-(n+1)]}{2}$$
$$x_{2_{odd}}[n]=\frac{\alpha^nu[n-1]-\alpha^{-n}u[-(n+1)]}{2}$$
\subsection*{(c)}

$$x_{3_{even}}[n]=\frac{n\alpha^nu[n-1]-n\alpha^{-n}u[-(n+1)]}{2}$$
$$x_{2_{even}}[n]=\frac{\alpha^nu[n-1]+n\alpha^{-n}u[-(n+1)]}{2}$$
\subsection*{(d)}
$$x_{4_{even}}[n]=x_4[n]=\alpha^{|n|}$$
$$x_{4_{odd}}[n]=0$$
\section*{Problem 2}
\subsection*{(a)}
False, for instance, $x_n[n]=\cos(\pi n)$ is a power signal since it has a bounded 
power of $p(x)=\frac{1}{2}\sum_{n=0}^1|\cos(\pi n)|^2=1$. while the energy is 
$\sum_{n=-\infty}^\infty|\cos(\pi n)|^2=\infty$.
\subsection*{(b)}
True, for an energy sequence we have that $\sum_{n=-\infty}^\infty|x_n[n]|^2=e_x<\infty$,
therefore for the power of the signal we have in general
$$P_x=\lim_{N\rightarrow\infty}\frac{1}{2N+1}\sum_{n=-N}^N|x_n[n]|^2$$
therefore
\begin{align*}
    P_x&=\lim_{N\rightarrow\infty}\frac{1}{2N+1}e_x\\
    &=e_x\lim_{N\rightarrow\infty}\frac{1}{2N+1}\\
    &=0
\end{align*}
\subsection*{(c)}
True, first we start by proving that the energy of a signal $x[n-1]$ is the 
same as the energy of $x[n]$, $e_x$. We have that the energy of $x[n-1]$, $e_{x-1}$ is 
$$e_{x-1}=\sum_{n=-\infty}^\infty|x[n-1]|^2$$
Letting k=n-1 we have that
\begin{align*}
    \sum_{n=-\infty}^\infty|x[n-1]|^2&=\sum_{k=-\infty-1}^{\infty+1}|x[k]|^2\\
    &=\sum_{k=-\infty}^{\infty}|x[k]|^2\\
    &=e_x
\end{align*}
therefore we have that 
\begin{align*}
e_x-e_{x-1}&=\sum_{n=-\infty}^\infty|x[n]|^2-\sum_{k=-\infty-1}^{\infty+1}|x[k]|^2\\
e_x-e_x&=|x[-\infty]|^2+|x[\infty]|^2\\
0&=|x[-\infty]|^2+|x[\infty]|^2
\end{align*}
Since $|x[-\infty]|^2\geq0$ and $|x[\infty]|^2\geq0$ we have that $x[n]=0$ as $n\to \infty$
\subsection*{(d)}
False, let $x[n]=\sqrt{|n|}$, then we have that the power of the signal $P_x$ is 
\begin{align*}
    P_x&=\lim_{N\rightarrow\infty}\frac{1}{2N+1}\sum_{n=-N}^N|x_n[n]|^2\\
    &=\lim_{N\rightarrow\infty}\frac{1}{2N+1}\sum_{n=-N}^N\left(\sqrt{|n|}\right)^2\\
    &=\lim_{N\rightarrow\infty}\frac{1}{2N+1}\sum_{n=-N}^N|n|\\
    &=\lim_{N\rightarrow\infty}\frac{1}{2N+1}\left(1+2\sum_{n=1}^Nn\right)\\
    &=\lim_{N\rightarrow\infty}\frac{1}{2N+1}\left(1+2\frac{N(N+1)}{2}\right)\\
    &=\lim_{N\rightarrow\infty}\frac{N(N+1)+1}{2N+1}\\
    &=\infty
\end{align*}
\section*{Problem 3}
For System 1 let $x[n]\leq B <\infty$ for all $n$, we have
$$y[n]=\log(|x[n-1]|)$$
since $|x[n-1]|\leq B$ for all $n$ and since $\log(x)$ is an convex function
$$|y[n]|=|\log(|x[n-1]|)|\leq |\log(B)|<\infty$$
Therefore System I is BIBO stable. To test for time invariant, let 
the input $x[n-k]$ corresponds to the output $y_1$, then we have that
$$y_1[n]=\log(|x[(n-1)-k]|)=y[n-k]$$
therefore System I is time invariant.\\\\
For System II let $|x[n]|\leq B <\infty$ for all $n$, we have
$$y[n]=e^{x[2n]}$$
since $x[n]\leq B$ for all $n$ and since $e^x$ is an convex function, we have
$$|y[n]|=e^{x[2n]}\leq e^B<\infty$$
Therefore System II is BIBO stable. To test for time invariant, let
the input $x[n-k]$ corresponds to the output $y_2$, then we have that
$$y_2[n]=e^{x[2n-k]}$$
this is not equal to 
$$y[n-k]=e^{x[2(n-k)]}$$
therefore System II is time variant.\\\\
Thus we have that statement (a) is correct.
\section*{Problem 4}
\subsection*{(a)}
The system is not linear, let 
$x_1[n]$ corresponds to the output $y_1$, and $x_2[n]$ corresponds to the output $y_2$, then we have that, 
the output $y'[n]$ corresponding to the input $\alpha x_1[n]+\beta x_2[n]$ is
$$y'[n]=\ln(|\alpha x_1[n-1]+\beta x_2[n-1]|+1)$$
which is not equal to $\alpha y_1[n]+\beta y_2[n]$\\\\
The system is time invariant, let $x[n-k]$ corresponds to the output $y_3[n]$, then we have that
$$y_3[n]=\ln(|x[n-k-1]|+1)=y[n-k]$$
The system is casual since it only depends on the input values at index $n$.\\\\
The system is BIBO stable, since given an input $x[n]\leq B<\infty$ for all $n$, we have that
$$|y[n]|=|\ln(|x[n-1]|+1)|\leq |\ln(|B|+1)|<\infty$$
The system is relaxed since given an input that goes to $0$ as $n\to\infty$, the output
also goes to $\ln(1)=0$.
\subsection*{(b)}
The system is linear, let 
$x_1[n]$ corresponds to the output $y_1$, and $x_2[n]$ corresponds to the output $y_2$, then we have that, 
the output $y'[n]$ corresponding to the input $\alpha x_1[n]+\beta x_2[n]$ is
$$y'[n]=y'[n-1]+x[n]$$
$$y'[n]=y'[-1]+\sum_{n=0}^{n}\alpha x_1[n]+\beta x_2[n]$$
$$y'[n]=\alpha\sum_{k=0}^{n}x_1[k]+\beta\sum_{k=0}^{n}x_1[k]$$
which is equal to $\alpha y_1[n]+\beta y_2[n]$\\\\
The system is time variant, let $x[n-k]$ corresponds to the output $y_3[n]$, then we have that
\begin{align*}
    y_3[n]&=y_3[n-1]+x[n-k]\\
    &=\sum_{m=0}^{n}x[m-k]\\
    &=\sum_{m-=k}^{n-k}x[m]
\end{align*}
This is diffrent from 
\begin{align*}
    y[n-k]&=y[n-k-1]+x[n-k]\\
    &=\sum_{m=0}^{n-k}x[m]
\end{align*}
Therefore the system is time variant.\\\\
The system is casual since it only depends on the input values at index $n$.\\\\
The system is not BIBO stable, since given an input $x[n]\leq B<\infty$ for all $n$, we have that
$$|y[n]|=|y[n-1]+x[n]|\leq \sum_{m=0}^{n}B =nB$$
This goes to $\infty$ as $n\to\infty$ so the system is not BIBO stable.\\\\
The system is not relaxed since it accumulates the input values over all indexs $0\leq n$ so
the output will not go to $0$ as the input goes to $0$ as $n\to\infty$.
\subsection*{(c)}
The system is non linear, let 
$x_1[n]$ corresponds to the output $y_1$, and $x_2[n]$ corresponds to the output $y_2$, then we have that, 
the output $y'[n]$ corresponding to the input $\alpha x_1[n]+\beta x_2[n]$ is
$$y'[n]=y'[n-1]+x[n]$$
$$y'[n]=y'[-1]+\sum_{n=0}^{n}\alpha x_1[n]+\beta x_2[n]$$
$$y'[n]=1+\alpha\sum_{k=0}^{n}x_1[k]+\beta\sum_{k=0}^{n}x_1[k]$$
which is not equal to $y_1[n]+y_2[n]=(\alpha+\beta)+\alpha\sum_{k=0}^{n}x_1[k]+\beta\sum_{k=0}^{n}x_1[k]$\\\\
The system is time variant, let $x[n-k]$ corresponds to the output $y_3[n]$, then we have that
\begin{align*}
    y_3[n]&=y_3[n-1]+x[n-k]\\
    &=1+\sum_{m=0}^{n}x[m-k]\\
    &=1+\sum_{m-=k}^{n-k}x[m]
\end{align*}
This is diffrent from 
\begin{align*}
    y[n-k]&=y[n-k-1]+x[n-k]\\
    &=1+\sum_{m=0}^{n-k}x[m]
\end{align*}
Therefore the system is time variant.\\\\
The system is casual since it only depends on the input values at index $n$.\\\\
The system is not BIBO stable, since given an input $x[n]\leq B<\infty$ for all $n$, we have that
$$|y[n]|=|y[n-1]+x[n]|\leq 1+\sum_{m=0}^{n}B =nB$$
This goes to $\infty$ as $n\to\infty$ so the system is not BIBO stable.\\\\
The system is not relaxed since it accumulates the input values over all indexs $0\leq n$ so
the output will not go to $0$ as the input goes to $0$ as $n\to\infty$.
\subsection*{(d)}
The system is non linear, let 
$x_1[n]$ corresponds to the output $y_1$, and $x_2[n]$ corresponds to the output $y_2$, then we have that, 
the output $y'[n]$ corresponding to the input $\alpha x_1[n]+\beta x_2[n]$ is
$$y'[n]=2+\alpha x_1[n]+\beta x_2[n]$$
which is not equal to $\alpha y_1[n]+\beta y_2[n]=2(\alpha+\beta)+\alpha x_1[n]+\beta x_2[n]$\\\\

The system is time invariant, let $x[n-k]$ corresponds to the output $y_3[n]$, then we have that
\begin{align*}
    y_3[n]&=2+x[n-k]
\end{align*}
Therefore the system is time invariant since $y[n-k]=2+x[n-k]$.\\\\
The system is casual since it only depends on the input values at index $n$.\\\\
The system is BIBO stable, since given an input $x[n]\leq B<\infty$ for all $n$, we have that
$$|y[n]|=|2+x[n]|\leq 2+B < \infty$$
The system is relaxed since given an input that goes to $0$ as $n\to\infty$, the output
also goes to $0$ since it only depends on the input at index $n$.
\section*{Problem 5}
\subsection*{(a)}

This system is stable if $h[n]$ is BIBO stable if it was bounded.
So if $|a|<1$ then the output is bounded, so
the condition for stability is that $\boxed{|a|\geq1}$.\\\\
\subsection*{(b)}

This system is stable if $h[n]$ is BIBO stable if it was bounded. Therefore this system is stable for $\boxed{-\infty<a<\infty}$.
\subsection*{(c)}
The system is stable if either the geometric series converges, so $\boxed{|r|<1}$ is $\boxed{\omega_0\neq0}$, or 
if $\sin[n\omega_0]=0$ for all $n$, so if $\boxed{\omega_0=0}$
\subsection*{(d)}
The system will be stable if $\boxed{|a|<1}$
\subsection*{(e)}
The system is stable if $-\infty<K<\infty$.
\end{document}
