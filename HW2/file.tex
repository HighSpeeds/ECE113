\include{"../preamble.tex"}
\title{ECE 113 HW 2}
\begin{document}
\maketitle
\section*{Problem 1}
\subsection*{(a)}
$$x_{1_{even}}[n]=\frac{u(n-3)+u(-(n+3))}{2}$$
$$x_{1_{odd}}[n]=\frac{u(n-3)-u(-(n+3))}{2}$$
\subsection*{(b)}
$$x_{2_{even}}[n]=\frac{\alpha^nu[n-1]+\alpha^{-n}u[-(n+1)]}{2}$$
$$x_{2_{odd}}[n]=\frac{\alpha^nu[n-1]-\alpha^{-n}u[-(n+1)]}{2}$$
\subsection*{(c)}

$$x_{3_{even}}[n]=\frac{n\alpha^nu[n-1]-n\alpha^{-n}u[-(n+1)]}{2}$$
$$x_{2_{even}}[n]=\frac{\alpha^nu[n-1]+n\alpha^{-n}u[-(n+1)]}{2}$$
\subsection*{(d)}
$$x_{4_{even}}[n]=x_4[n]=\alpha^{|n|}$$
$$x_{4_{odd}}[n]=0$$
\section*{Problem 2}
\subsection*{(a)}
False, for instance, $x_n[n]=\cos(\pi n)$ is a power signal since it has a bounded 
power of $p(x)=\frac{1}{2}\sum_{n=0}^1|\cos(\pi n)|^2=1$. while the energy is 
$\sum_{n=-\infty}^\infty|\cos(\pi n)|^2=\infty$.
\subsection*{(b)}
True, for an energy sequence we have that $\sum_{n=-\infty}^\infty|x_n[n]|^2=e_x<\infty$,
therefore for the power of the signal we have in general
$$P_x=\lim_{N\rightarrow\infty}\frac{1}{2N+1}\sum_{n=-N}^N|x_n[n]|^2$$
therefore
\begin{align*}
    P_x&=\lim_{N\rightarrow\infty}\frac{1}{2N+1}e_x\\
    &=e_x\lim_{N\rightarrow\infty}\frac{1}{2N+1}\\
    &=0
\end{align*}
\subsection*{(c)}
True, first we start by proving that the energy of a signal $x[n-1]$ is the 
same as the energy of $x[n]$, $e_x$. We have that the energy of $x[n-1]$, $e_{x-1}$ is 
$$e_{x-1}=\sum_{n=-\infty}^\infty|x[n-1]|^2$$
Letting k=n-1 we have that
\begin{align*}
    \sum_{n=-\infty}^\infty|x[n-1]|^2&=\sum_{k=-\infty-1}^{\infty+1}|x[k]|^2\\
    &=\sum_{k=-\infty}^{\infty}|x[k]|^2\\
    &=e_x
\end{align*}
therefore we have that 
\begin{align*}
e_x-e_{x-1}&=\sum_{n=-\infty}^\infty|x[n]|^2-\sum_{k=-\infty-1}^{\infty+1}|x[k]|^2\\
e_x-e_x&=|x[-\infty]|^2+|x[\infty]|^2\\
0&=|x[-\infty]|^2+|x[\infty]|^2
\end{align*}
Since $|x[-\infty]|^2\geq0$ and $|x[\infty]|^2\geq0$ we have that $x[n]=0$ as $n\to \infty$
\subsection*{(d)}
True, let $x[n]=\sqrt{|n|}$, then we have that the power of the signal $P_x$ is 
\begin{align*}
    P_x&=\lim_{N\rightarrow\infty}\frac{1}{2N+1}\sum_{n=-N}^N|x_n[n]|^2\\
    &=\lim_{N\rightarrow\infty}\frac{1}{2N+1}\sum_{n=-N}^N\left(\sqrt{|n|}\right)^2\\
    &=\lim_{N\rightarrow\infty}\frac{1}{2N+1}\sum_{n=-N}^N|n|\\
    &=\lim_{N\rightarrow\infty}\frac{1}{2N+1}\left(1+2\sum_{n=1}^Nn\right)\\
    &=\lim_{N\rightarrow\infty}\frac{1}{2N+1}\left(1+2\frac{N(N+1)}{2}\right)\\
    &=\lim_{N\rightarrow\infty}\frac{N(N+1)+1}{2N+1}\\
    &=\infty
\end{align*}
\section*{Problem 3}
For System 1 let $x[n]\leq B <\infty$ for all $n$, we have
$$y[n]=\log(|x[n-1]|)$$
since $x[n-1]\leq B$ for all $n$ and since $\log(x)$ is an convex function
$$y[n]=\log(|x[n-1]|)\leq \log(B)<\infty$$
Therefore System I is BIBO stable. To test for time invariant, let 
the input $x[n-k]$ corresponds to the output $y_1$, then we have that
$$y_1[n]=\log(|x[(n-1)-k]|)=y[n-k]$$
therefore System I is time invariant.\\\\
For System II let $x[n]\leq B <\infty$ for all $n$, we have
$$y[n]=e^{x[2n]}$$
since $x[n]\leq B$ for all $n$ and since $e^x$ is an convex function, we have
$$y[n]=e^{x[2n]}\leq e^B<\infty$$
Therefore System II is BIBO stable. To test for time invariant, let
the input $x[n-k]$ corresponds to the output $y_2$, then we have that
$$y_2[n]=e^{x[2n-k]}$$
this is not equal to 
$$y[n-k]=e^{x[2(n-k)]}$$
therefore System II is time variant.\\\\
Thus we have that statement (a) is correct.

\end{document}
